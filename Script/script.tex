\documentclass{article}
\usepackage{hyperref}

\begin{document}
\noindent \underline{\bfseries Creator of \TeX: Donald Knuth, a well respected computer scientist}

\begin{itemize}


\item In his first volume of the book named \textquotedblleft The art of computer programming", it was typeset in 
metal-typesetting system with lots of hand work involved. 


\item But for the second volume, the publisher changed their printing technology into photo-typesetting. After he
got the \href{https://en.wikipedia.org/wiki/Galley_proof}{gallery proof} of his book. He was shocked because the quality of typesetting was awful.


\item The main issue was that the letter wasn't position accurately. And some words are more darker than others. And no quality control.


\item He said, 'Oh my god! I don't want to see my book like this!'


\item Then, he cancelled the final printing process.


\item During at Stanford, every year his community duty is to read new books and need to pick
which one is good for reading list to the following year students.


\item At that time, In 1997, he got a book by Patrick Winston though it was in gallery proof form. But, this book was in Artificial intelligence. And, that was typeset in a new way. 

\item This was done using the machine that was completely digital. It wasn't metal or photograph method but, it was a pixels.


\item At that time, even digital machine doesnot succeed to produce good quality.


\item So, this gave him the possibility to understand the document in-terms of pixels which encodes in 0 and 1. 


\item So, he got a clue that high quality printing is the matter of computer program. And saw in the form of computer problem

\item He was happy that, any person dealing with 0 and 1 was him.


\item Thus, he started with coding and implemented it to more than 500 pages of his book.


\item Finally, He shared his software within GNU license. 
Today, the current stable version of \TeX\hspace{0.1cm} is 3.14159265 that means to say, it is in nine version. 


\item There was alot of user using TeX at that time and they extended the macros of \TeX\hspace{0.1cm} and give some name. And the most popular one is \LaTeX. Thus, we have \textsf{*.tex} in \LaTeX \hspace{0.1cm} file extension. 

\noindent \underline{\bfseries Creator of \LaTeX: Leslie Lamport, computer scientist}
\vspace{0.2cm}

\item Date back to 1985, Leslie Lamport releases \LaTeX\hspace{0.1cm} to the modification of \TeX \hspace{0.1cm} encloses with markup language. Till now, the current version is \LaTeX2e. 

\item This is designed on the aim to have a easy to use document preparation system.

\end{itemize}

\end{document}